% !TeX spellcheck = fr_CA

\chapter*{Résumé}
\vspace{-2em}

\Beluga est un langage de programmation fonctionnelle et un assistant de preuve pour spécifier des systèmes formels dans \acs{LF} contextuel, une extension du \acl{LF}, et prouver mécaniquement des théorèmes à propos d'eux au moyen de programmes récursifs.
Pour faciliter le développement incrémental de preuves, l'environnement interactif de preuve \Harpoon est implémenté en tant que \acl{REPL} avec des fonctionnalités d'édition structurelle de programmes écrits dans \Beluga.
À cause de limitations architecturales dans l'implémentation de \Beluga et \Harpoon, le développement de preuves vertical ou dans le désordre peuvent entraîner l'invalidation des états de preuve et la génération de programmes incorrects.

Cette thèse présente les défis techniques et les solutions à l'implémentation cohérente d'édition structurelle avec la navigation d'un endroit à l'autre dans des preuves incomplètes.
Un analyseur syntaxique contextuel est réalisé pour supporter la définition d'opérateurs préfixes, infixes et postfixes par l'utilisateur.
Ensuite, la résolution de noms pour \Beluga est rectifiée avec l'implémentation d'un environnement de référencement uniforme pour l'indexation de programmes définis par rapport à classes disjointes de variables.
Enfin, ces systèmes sont intégrés dans \Harpoon pour garantir que l'état des identifiants visibles à n'importe quel endroit dans une preuve incomplète est cohérent avec son emplacement.

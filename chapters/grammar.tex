\chapter{\Beluga Grammar}

This appendix describes the \ac{CFG} for \Beluga and \Harpoon in their entirety as a specification for the parsing algorithm.
The lexical conventions are defined first, then ambiguous \acp{CFG} are presented for the various kinds of syntactic structures, and finally blurred \acp{CFG} are shown to illustrate how context-free parsing is done before disambiguation.

\section{Syntax}\label{section:syntax}

This section presents the syntax for \Beluga signatures using ambiguous \acp{CFG}.
It is intended to serve as a specification for validating the parser's implementation, as well as to assist in making changes to the parser.
Rewriting of the grammar for optimization, left recursion elimination, and disambiguation purposes is outside the scope of this presentation.
Section~\ref{section:resolving-grammar-ambiguities} presents a \ac{CFG} for \Beluga that is suitable for parsing with a separate stage for disambiguation.

The \ac{EBNF} metasyntax used in the sections hereafter follows this notation:

\begin{itemize}
\item $ (a)* $ represents the repeated application of $ a $ zero or more times.
\item $ (a)+ $ represents the repeated application of $ a $ one or more times.
\item $ [a] $ represents optionally applying $ a $.
\item $ a - b $ represents the application of $ a $ that cannot be the application of $ b $ if $ a $ and $ b $ are single-character length productions.
\end{itemize}

\subsection{Comments}\label{section:comments-lexical-convention}

\Beluga supports three kinds of comments:

\begin{itemize}
\item
Line comments start with \texttt{\%} and end with a newline \texttt{\textbackslash n}.
They are treated as blank characters, and as such may be interspersed anywhere in the grammar.
\item
Block comments are enclosed in \texttt{\%\{} and \texttt{\}\%}.
They are treated as blank characters, and as such may be interspersed anywhere in the grammar.
\item
Documentation comments are enclosed in \texttt{\%\{\{} and \texttt{\}\}\%}.
The contents of documentation comments are parsed as \Markdown following the \href{https://spec.commonmark.org/0.29}{\Commonmark standard version 0.29}.
These are used in documentation generation to \HTML or \LaTeX.
Documentation comments may only appear before or after signature-level declarations.
\end{itemize}

\subsection{Keywords}\label{section:keywords-lexical-convention}

The following identifiers are reserved as keywords in \Beluga, and cannot be used otherwise.

\begin{multicols}{4}
\begin{itemize}[label={}, nolistsep]
\item \texttt{LF}
\item \texttt{and}
\item \texttt{impossible}
\item \texttt{case}
\item \texttt{rec}
\item \texttt{fn}
\item \texttt{fun}
\item \texttt{mlam}
\item \texttt{if}
\item \texttt{then}
\item \texttt{else}
\item \texttt{of}
\item \texttt{schema}
\item \texttt{some}
\item \texttt{block}
\item \texttt{world}
\item \texttt{module}
\item \texttt{struct}
\item \texttt{end}
\item \texttt{trust}
\item \texttt{total}
\item \texttt{prop}
\item \texttt{type}
\item \texttt{ctype}
\item \texttt{inductive}
\item \texttt{coinductive}
\item \texttt{stratified}
\item \texttt{typedef}
\item \texttt{proof}
\item \texttt{by}
\item \texttt{as}
\item \texttt{suffices}
\item \texttt{toshow}
\end{itemize}
\end{multicols}

Additionally, the following characters are reserved and cannot be used anywhere in identifiers.

{\setlength{\columnseprule}{0.4pt}
\begin{multicols}{3}
\begin{itemize}[label={}, nolistsep]
\item \texttt{.}
\item \texttt{…}
\item \texttt{,}
\item \texttt{:}
\item \texttt{;}
\item \texttt{\%}
\item \texttt{|}
\item \texttt{"}
\item \texttt{\textbackslash}
\item \texttt{( )}
\item \texttt{[ ]}
\item \texttt{\{ \}}
\item \texttt{< >}
\item \texttt{⊢}
\end{itemize}
\end{multicols}
}

\subsection{Lexical Conventions}\label{section:lexical-convention}

\Beluga programs must be encoded in \textsc{UTF-8}.
The nonterminal \synt{white-space} corresponds to the Unicode property \texttt{White\_Space} as described in the Unicode Standard Version 15.0.0 Annex \#44, \href{https://www.unicode.org/reports/tr44/tr44-30.html#White_Space}{\textit{Unicode Character Database}}.
The nonterminal \synt{any} stands for any unicode character.

\begin{grammar}
<forward-arrow> ::=
  `->'
| `→'

<backward-arrow> ::=
  `<-'
| `←'

<thick-forward-arrow> ::=
  `=>'
| `⇒'

<turnstile> ::=
  `|-'
| `⊢'

<turnstile-hash> ::= <turnstile>`#'

<dots> ::=
  `..'
| `…'

<non-zero-digit> ::= `1'..`9'

<digit> ::=
  `0'
| <non-zero-digit>

<integer> ::= `0' | <non-zero-digit> <digit>*

<ascii-control> ::= `\\000' .. `\\031' | `\\127'

<reserved-character> ::= `.' | `,' | `:' | `;' | `\%' | `|' | `"' | `\\' | `(' | `)' | `[' | `]' \alt `{' | `}' | `<' | `>' | `⊢'

<identifier-continue> ::= <any> - (<ascii-control> | <white-space> | <reserved-character>)

<identifier-start> ::= <identifier-continue> - (<digit> | `$' | `#')

<identifier> ::= <identifier-start> <identifier-continue>*

<dot-integer> ::= `.'<integer>

<dot-identifier> ::= `.'<identifier>

<hash-identifier> ::= `#'<identifier>

<dollar-identifier> ::= `$'<identifier>
\end{grammar}

\subsection{Utilities}

\begin{grammar}
<omittable-identifier> ::=
  `_'
| <identifier>

<omittable-hash-identifier> ::=
  `#_'
| <hash-identifier>

<omittable-dollar-identifier> ::=
  `$_'
| <dollar-identifier>

<qualified-identifier> ::= <identifier> <dot-identifier>*

<dot-qualified-identifier> ::= <dot-identifier>+

<meta-object-identifier> ::= <identifier>
\alt <hash-identifier>
\alt <dollar-identifier>

<omittable-meta-object-identifier> ::= <omittable-identifier>
\alt <omittable-hash-identifier>
\alt <omittable-dollar-identifier>

<precedence> ::= <integer>

<associativity> ::=
  `left'
| `right'
| `none'
\end{grammar}

\subsection{Grammar for \acs{LF}}\label{section:syntax-lf}

The following is the grammar for the definition of \LF kinds, types and terms.
These syntactic constructs form a weak representational function space without case analysis or recursion.

\begin{grammar}
<lf-kind> ::= `{' <omittable-identifier> `:' <lf-type> `}' <lf-kind>
\alt <lf-type> <forward-arrow> <lf-kind>
\alt `type'
\alt `(' <lf-kind> `)'

<lf-type> ::= <qualified-identifier>
\alt `{' <omittable-identifier> `:' <lf-type> `}' <lf-type>
\alt <lf-type> <forward-arrow> <lf-type>
\alt <lf-type> <backward-arrow> <lf-type>
\alt <qualified-identifier> <lf-term>
\alt <lf-term> <qualified-identifier> <lf-term>
\alt <lf-term> <qualified-identifier>
\alt `(' <lf-type> `)'

<lf-term> ::= <identifier>
\alt <qualified-identifier>
\alt `\\' `(' <omittable-identifier> `:' <lf-type> `)' `.' <lf-term>
\alt `\\' <omittable-identifier> `.' <lf-term>
\alt <lf-term> <lf-term>+
\alt <qualified-identifier> <lf-term>
\alt <lf-term> <qualified-identifier> <lf-term>
\alt <lf-term> <qualified-identifier>
\alt `_'
\alt <lf-term> `:' <lf-type>
\alt `(' <lf-term> `)'
\end{grammar}

\subsection{Grammar for Contextual \acs{LF}}\label{section:syntax-contextual-lf}

The following is the grammar for the extension of \LF with substitutions, contexts and pattern-matching.

\begin{grammar}
<clf-type> ::= <qualified-identifier>
\alt `{' <omittable-identifier> `:' <clf-type> `}' <clf-type>
\alt <clf-type> <forward-arrow> <clf-type>
\alt <clf-type> <backward-arrow> <clf-type>
\alt `block' `(' <identifier> `:' <clf-type> (`,' <identifier> `:' <clf-type>)+ `)'
\alt `block' <identifier> `:' <clf-type> (`,' <identifier> `:' <clf-type>)+
\alt `block' `(' <clf-type> `)'
\alt `block' <clf-type>
\alt <qualified-identifier> <clf-term>
\alt <clf-term> <qualified-identifier> <clf-term>
\alt <clf-term> <qualified-identifier>
\alt `(' <clf-type> `)'

<clf-term> ::= <identifier>
\alt <hash-identifier>
\alt <qualified-identifier>
\alt <clf-term> <clf-substitution>
\alt `\\' `(' <omittable-identifier> `:' <clf-type> `)' `.' <clf-term>
\alt `\\' <omittable-identifier> `.' <clf-term>
\alt <clf-term> <clf-term>+
\alt <qualified-identifier> <clf-term>
\alt <clf-term> <qualified-identifier> <clf-term>
\alt <clf-term> <qualified-identifier>
\alt `_'
\alt `?'[<identifier>]
\alt `<' <clf-term> (`;' <clf-term>)* `>'
\alt <clf-term> <dot-integer>
\alt <clf-term> <dot-identifier>
\alt <clf-term> `:' <clf-type>
\alt `(' <clf-term> `)'

<clf-pattern> ::= <identifier>
\alt <hash-identifier>
\alt <qualified-identifier>
\alt `_'
\alt `<' <clf-pattern> (`;' <clf-pattern>)* `>'
\alt <clf-pattern> <dot-integer>
\alt <clf-pattern> <dot-identifier>\\
\alt `\\' `(' <omittable-identifier> `:' <clf-type> `)' `.' <clf-pattern>
\alt `\\' <omittable-identifier> `.' <clf-pattern>
\alt <clf-pattern> `[' <clf-substitution> `]'
\alt <clf-pattern> <clf-pattern>+
\alt <qualified-identifier> <clf-pattern>
\alt <clf-pattern> <qualified-identifier> <clf-pattern>
\alt <clf-pattern> <qualified-identifier>
\alt <clf-pattern> `:' <clf-type>
\alt `(' <clf-pattern> `)'

<clf-substitution> ::= [`^']
\alt <dollar-identifier> (`,' <clf-term>)*
\alt <dollar-identifier> `[' <clf-substitution> `]' (`,' <clf-term>)*
\alt `..' (`,' <clf-term>)*
\alt <clf-term> (`,' <clf-term>)*

<clf-substitution-pattern> ::= [`^']
\alt <dollar-identifier> (`,' <clf-pattern>)*
\alt <dollar-identifier> `[' <clf-substitution> `]'(`,' <clf-pattern>)*
\alt `..' (`,' <clf-pattern>)*
\alt <clf-pattern> (`,' <clf-pattern>)*

<clf-context> ::= [`^']
\alt <omittable-identifier> (`,' <identifier> [`:' <clf-type>])*
\alt <identifier> [`:' <clf-type>] (`,' <identifier> [`:' <clf-type>])*

<clf-context-pattern> ::= [`^']
\alt <omittable-identifier> (`,' <identifier> `:' <clf-type>)*
\alt <identifier> `:' <clf-type> (`,' <identifier> `:' <clf-type>)*
\end{grammar}

\subsection{Grammar for the Meta-Level}\label{section:syntax-meta-level}

The following is the grammar for meta-level types (meta-types) and objects (meta-objects).
Meta-types classify meta-objects.
Meta-objects are contextual \LF terms, substitutions and contexts.

Meta-types and meta-objects may be embedded in the computation level by denoting them with boxes.
Modifying the box annotation by prefixing it with \syntax{\#} or \syntax{\$} syntactically forces it to be a specific variant of meta-type or meta-object.
Substitution types and objects are required to be annotated with a \syntax{\$}-box to disambiguate them from contextual types and terms respectively.
When paired with an identifier in a declaration such as in a meta-context, the prefixes of the identifier and the boxed meta-type must match.
Exceptionally, schema meta-types are not boxed.

\begin{grammar}
<schema> ::= <qualified-identifier>
\alt <schema> `+' <schema>
\alt [`some' `[' <identifier> `:' <lf-type> (`,' <identifier> `:' <lf-type>)* `]'] `block' `(' <identifier> `:' <lf-type> (`,' <identifier> `:' <lf-type>)* `)'
\alt [`some' `[' <identifier> `:' <lf-type> (`,' <identifier> `:' <lf-type>)* `]'] `block' <identifier> `:' <lf-type> (`,' <identifier> `:' <lf-type>)
\alt [`some' `[' <identifier> `:' <lf-type> (`,' <identifier> `:' <lf-type>)* `]'] `block' <lf-type>
\alt [`some' `[' <identifier> `:' <lf-type> (`,' <identifier> `:' <lf-type>)* `]'] `block' <lf-type>

<meta-type> ::= <qualified-identifier>
\alt `(' <clf-context> <turnstile> <clf-type> `)'
\alt `#(' <clf-context> <turnstile> <clf-type> `)'
\alt `$(' <clf-context> <turnstile> <clf-context> `)'
\alt `$(' <clf-context> <turnstile-hash> <clf-context> `)'
\alt `[' <clf-context> <turnstile> <clf-type> `]'
\alt `#[' <clf-context> <turnstile> <clf-type> `]'
\alt `$[' <clf-context> <turnstile> <clf-context> `]'
\alt `$[' <clf-context> <turnstile-hash> <clf-context> `]'

<meta-object> ::= `[' <clf-context> `]'
\alt `[' <clf-context> <turnstile> <clf-term> `]'
\alt `$[' <clf-context> <turnstile> <clf-substitution> `]'
\alt `$[' <clf-context> <turnstile-hash> <clf-substitution> `]'

<meta-pattern> ::= `[' <clf-context-pattern> `]'
\alt `[' <clf-context-pattern> <turnstile> <clf-pattern> `]'
\alt `$[' <clf-context-pattern> <turnstile> <clf-substitution-pattern> `]'
\alt `$[' <clf-context-pattern> <turnstile-hash> <clf-substitution-pattern> `]'

<meta-context> ::= [`^']
\alt <meta-object-identifier> [`:' <meta-type>]\\
(`,' <meta-object-identifier> [`:' <meta-type>])*
\end{grammar}

\subsection{Grammar for Computations}\label{section:syntax-computations}

The following is the grammar for the computational-level types, expressions and patterns.
Atomic patterns are computation-level patterns that may appear as whitespace-delimited lists in pattern-matching functions.

\begin{grammar}
<comp-kind> ::= `{' <omittable-meta-object-identifier> : <meta-type> `}' <comp-kind>
\alt `(' <omittable-meta-object-identifier> : <meta-type> `)' <comp-kind>
\alt <meta-type> <forward-arrow> <comp-kind>
\alt `ctype'
\alt `(' <comp-kind> `)'

<comp-type> ::= `{' <omittable-meta-object-identifier> `:' <meta-type> `}' <comp-type>
\alt `(' <omittable-meta-object-identifier> `:' <meta-type> `)' <comp-type>
\alt <comp-type> <forward-arrow> <comp-type>
\alt <comp-type> <backward-arrow> <comp-type>
\alt <comp-type> (`*' <comp-type>)+
\alt <meta-type>
\alt <qualified-identifier> <meta-object>*
\alt <qualified-identifier> <meta-object>
\alt <meta-object> <qualified-identifier> <meta-object>
\alt <meta-object> <qualified-identifier>
\alt <qualified-identifier> <meta-object>*
\alt <qualified-identifier> <meta-object>
\alt <meta-object> <qualified-identifier> <meta-object>
\alt <meta-object> <qualified-identifier>
\alt `(' <comp-type> `)'

<comp-expression> ::= <identifier>
\alt <qualified-identifier>
\alt <qualified-identifier>
\alt `fn' <omittable-identifier>+ <thick-forward-arrow> <comp-expression>
\alt `fun' [`|'] <comp-cofunction-branch> (`|' <comp-cofunction-branch>)*
\alt `mlam' <omittable-meta-object-identifier>+\\
<thick-forward-arrow> <comp-expression>
\alt `let' [<comp-pattern-meta-context>] <comp-pattern>\\
`=' <comp-expression> `in' <comp-expression>
\alt <meta-object>
\alt `impossible' <comp-expression>
\alt `case' <comp-expression> [`----not'] `of' [`|']\\
<comp-case-branch> (`|' <comp-case-branch>)*
\alt `(' <comp-expression> (`,' <comp-expression>)+ `)'
\alt `?' [<identifier>]
\alt `_'
\alt <comp-expression> <comp-expression>+
\alt <qualified-identifier> <comp-expression>
\alt <comp-expression> <qualified-identifier> <comp-expression>
\alt <comp-expression> <qualified-identifier>
\alt <comp-expression> <dot-qualified-identifier>
\alt <comp-expression> `:' <comp-type>
\alt `(' <comp-expression> `)'

<comp-case-branch> ::= [<comp-pattern-meta-context>] <comp-pattern>\\
<thick-forward-arrow> <comp-expression>

<comp-cofunction-branch> ::= ([<comp-pattern-meta-context>] <comp-copattern>)+\\
<thick-forward-arrow> <comp-expression>

<comp-pattern-meta-context> ::= (`{' <meta-object-identifier> [`:' <meta-type>] `}')+

<comp-pattern> ::= <identifier>
\alt <qualified-identifier>
\alt <meta-pattern>
\alt `(' <comp-pattern> (`,' <comp-pattern>)+ `)'
\alt <comp-pattern> <comp-pattern>+
\alt <qualified-identifier> <comp-pattern>
\alt <comp-pattern> <qualified-identifier> <comp-pattern>
\alt <comp-pattern> <qualified-identifier>
\alt <comp-pattern> `:' <comp-type>
\alt `_'
\alt `(' <comp-pattern> `)'

<comp-pattern-atomic> ::= <identifier>
\alt <qualified-identifier>
\alt <meta-pattern>
\alt `(' <comp-pattern> (`,' <comp-pattern>)+ `)'
\alt `_'
\alt `(' <comp-pattern> `)'

<comp-copattern> ::= <dot-qualified-identifier> <comp-pattern-atomic>*
\alt <comp-pattern-atomic>

<comp-context> ::= [`^']
\alt <identifier> [`:' <comp-type>] (`,' <identifier> [`:' <comp-type>])*
\end{grammar}

\subsection{Grammar for \Harpoon's \acs{REPL}}\label{section:syntax-harpoon-repl}

The following is the grammar for the commands used in the \Harpoon \ac{REPL}.

\begin{grammar}
<harpoon-command> ::= `intros' <identifier>*
\alt `split' <comp-expression>
\alt `invert' <comp-expression>
\alt `impossible' <comp-expression>
\alt `msplit' <identifier>
\alt `solve' <comp-expression>
\alt `by' <comp-expression> `as' <identifier> [<boxity>]
\alt `type' <comp-expression>
\alt `suffices' `by' <comp-expression>\\
`toshow' [(`_' | <comp-type>) (`,' (`_' | <comp-type>))*]
\alt `unbox' <comp-expression> `as' <identifier>
\alt `strengthen' <comp-expression> `as' <identifier>
\alt `toggle-automation' <automation-kind> [<automation-change>]
\alt `rename' (`comp' | `meta') <identifier> <identifier>
\alt `defer'
\alt `select' <qualified-identifier>
\alt `info' `theorem' <qualified-identifier>
\alt `theorem' <theorem-subcommand>
\alt `session' <session-subcommand>
\alt `subgoal' <subgoal-subcommand>
\alt `undo'
\alt `redo'
\alt `history'
\alt `help'
\alt `save'

<boxity> ::= `boxed' | `unboxed' | `strenghtened'

<automation-kind> ::= `auto-intros' | `auto-solve-trivial'

<automation-change> ::= `on' | `off' | `toggle'

<theorem-subcommand> ::= `list' | `defer' | `dump-proof' <file-path> | `show-ihs' \alt `show-proof'

<session-subcommand> ::= `list' | `defer' | `create' | `serialize'

<subgoal-subcommand> ::= `list' | `defer'
\end{grammar}

\subsection{Grammar for \Harpoon's Proof Scripts}\label{section:syntax-harpoon-proof-scripts}

The following is the grammar for the serialized form of \Harpoon interactive sessions.

\begin{grammar}
<harpoon-proof> ::= `?'[<identifier>]
\alt <harpoon-command> `;' <harpoon-proof>
\alt <harpoon-directive>

<harpoon-directive> ::= `intros' <harpoon-hypothetical>
\alt `solve' <comp-expression>
\alt `split' <comp-expression> `as' <harpoon-case>*
\alt `impossible' <comp-expression>
\alt `suffices' `by' <comp-expression> `toshow' <comp-type> `{' <harpoon-proof> `}'

<harpoon-case> ::= `case' <harpoon-case-label> `:' <harpoon-hypothetical>

<harpoon-case-label> ::= `extended' `by' <integer>
\alt `empty' `context'
\alt <qualified-identifier>
\alt `#' [<integer>] [<dot-integer>]
\alt `head' `variable'

<harpoon-hypothetical> ::= <meta-context> `|' <comp-context> `;' <harpoon-proof>
\end{grammar}

\subsection{Grammar for Pragmas}\label{section:syntax-pragmas}

The following is the grammar for pragmas that adjust parameters for parsing, elaboration or printing.
In particular, pragmas allow the user to choose a notation for operators.

\begin{grammar}
<pragma> ::= `----prefix' <qualified-identifier> [<precedence>] `.'
\alt `----infix' <qualified-identifier> [<precedence>] [<associativity>] `.'
\alt `----postfix' <qualified-identifier> [<precedence>] `.'
\alt `----name' <qualified-identifier> <identifier> [<identifier>] `.'
\alt `----not'
\alt `----assoc <associativity> `.'
\alt `----open <qualified-identifier> `.'
\alt `----abbrev <qualified-identifier> <identifier> `.'
\alt `----query' (`*' | <integer>) (`*' | <integer>) [<identifier>] `:' <clf-term> `.'

<global-pragma> ::= `----nostrenghten' | `----coverage' | `----warncoverage'
\end{grammar}

\subsection{Grammar for \Beluga Signatures}\label{section:syntax-signature}

The following is the grammar for toplevel code sections in \Beluga.
Constant declarations are arranged as lists, and may appear in modules for namespacing.
This includes the declaration of \LF type-level and term-level constants, computation-level type
That is, type constants, term constants, programs and proofs are arranged as lists of declarations, and may appear in modules.

\begin{grammar}
<signature> ::= <global-pragma>* <entry>*

<entry> ::= <declaration> | <pragma> | <documentation-comment>

<declaration> ::= <module-declaration>
\alt <lf-type-constants-declaration>
\alt <lf-type-constant-plain-declaration>
\alt <lf-term-constant-plain-declaration>
\alt <comp-type-constants-declaration>
\alt <mutually-recursive-programs-declaration>
\alt <schema-declaration>
\alt <typedef-declaration>
\alt <let-declaration>

<module-declaration> ::= `module' <identifier> `=' `struct' <entry>* `end' [`;']

<lf-type-constants-declaration> ::= `LF' <lf-type-constant-declaration>\\
(`and' [`LF'] <lf-type-constant-declaration>)* `;'

<lf-type-constant-declaration> ::= <identifier> `:' <lf-kind>\\
`=' [`|'] [<lf-term-constant-declaration>] (`|' <lf-term-constant-declaration>)*

<lf-term-constant-declaration> ::= <identifier> `:' <lf-type>

<lf-type-constant-plain-declaration> ::= <identifier> `:' <lf-kind> `.'

<lf-term-constant-plain-declaration> ::= <identifier> `:' <lf-type> `.'

<comp-type-constants-declaration> ::= <comp-type-constant-declaration>\\
(`and' <comp-type-constant-declaration>)* `;'

<comp-type-constant-declaration> ::= <inductive-type-constant-declaration>
\alt <coinductive-type-constant-declaration>
\alt <stratified-type-constant-declaration>

<inductive-type-constant-declaration> ::= `inductive' <inductive-type-constant>

<inductive-type-constant> ::= <identifier> `:' <comp-kind>\\
`=' [`|'] [<inductive-expression-constant>] (`|' <inductive-expression-constant>)*

<inductive-expression-constant> ::= <identifier> `:' <comp-type>

<coinductive-type-constant-declaration> ::= `coinductive' <coinductive-type-constant>

<coinductive-type-constant> ::= <identifier> `:' <comp-kind>\\
`=' [`|'] [<coinductive-expression-constant>] (`|' <coinductive-expression-constant>)*

<coinductive-expression-constant> ::= <identifier> `:' <comp-type> `::' <comp-type>

<stratified-type-constant-declaration> ::= `stratified' <stratified-type-constant>

<stratified-type-constant> ::= <identifier> `:' <comp-kind>\\
`=' [`|'] [<stratified-expression-constant>] (`|' <stratified-expression-constant>)*

<stratified-expression-constant> ::= <identifier> `:' <comp-type>

<proof-declaration> ::= `proof' <identifier> `:' <comp-type> `=' [<totality-declaration>] <harpoon-proof>

<program-declaration> ::= `rec' <identifier> `:' <comp-type> `=' [<totality-declaration>] <comp-expression>

<mutually-recursive-programs-declaration> ::= (<proof-declaration> | <program-declaration>) (`and' (<proof-declaration> | <program-declaration>))* `;'

<totality-declaration> ::= `/' `total' `/'
\alt `/' `total' <named-totality-order> `(' <identifier> <omittable-identifier>+ `)' `/'
\alt `/' `total' <numeric-totality-order> `/'
\alt `/' `trust' `/'

<named-totality-order> ::= <named-argument-totality-order>
\alt <named-lexical-totality-order>
\alt <named-simultaneous-totality-order>

<named-argument-totality-order> ::= <identifier>

<named-lexical-totality-order> ::= `{' <named-totality-order>+ `}'

<named-simultaneous-totality-order> ::= `[' <named-totality-order>+ `]'

<numeric-totality-order> ::= <numeric-argument-totality-order>
\alt <numeric-lexical-totality-order>
\alt <numeric-simultaneous-totality-order>

<numeric-argument-totality-order> ::= <integer>

<numeric-lexical-totality-order> ::= `{' <numeric-totality-order>+ `}'

<numeric-simultaneous-totality-order> ::= `[' <numeric-totality-order>+ `]'

<schema-declaration> ::= `schema' <identifier> `=' <schema> `;'

<typedef-declaration> ::= `typedef' <identifier> `:' <comp-kind> `=' <comp-type> `;'

<let-declaration> ::= `let' <identifier> `:' <comp-type> `=' <comp-expression> `;'

<documentation-comment> ::= `%{{' <any>* `}}%'
\end{grammar}

\section{Resolving Syntactic Ambiguities}\label{section:resolving-grammar-ambiguities}

This section highlights some syntactic ambiguities in the syntax for \Beluga as presented in section~\ref{section:syntax}, and presents strategies for resolving them.
Ambiguities in the syntax include:
\begin{itemize}
\item Fully qualified identifiers, projections, $\lambda$-terms, old-style \LF declarations and destructors use the same dot operator.
\item \LF kinds, types and terms share syntactic constructs for arrows, $ \Pi $'s and applications.
\item Contextual \LF types, terms and patterns share syntactic constructs for arrows and applications.
\item Computation kinds and types share syntactic constructs for arrows, $ \Pi $'s and applications.
\item Ambiguities associated with the usual operators in expression grammars with their associativities and precedences.
\item Ambiguities associated with user-defined operators with their fixities, associativities and precedences.
\end{itemize}

Resolution for ambiguities in a grammar is either stateless (context-free) using grammar rewriting, or stateful (context-sensitive or data-dependent) using semantic analysis data produced during or after parsing.

Ambiguities for the built-in operators in the syntax may be resolved statelessly by rewriting the grammar to be suitable for recursive-descent parsing.

The remainder of the ambiguities listed above must be resolved statefully since they require complete scope information to correctly resolve operator sorts and the parser settings defined by the user.
Furthermore, given that declarations in a \Beluga signature may be mutually recursive, resolving these ambiguities is best done in a separate phase after parsing.
Indeed, attempting to resolve these ambiguities in a top-down parser would require possibly unbounded lookaheads and more costly backtracking.
As such, a \Beluga signature is first parsed into an \ac{AST} with ambiguous nodes, and then a separate disambiguation phase elaborates that \ac{AST} to the external syntax.
By design, the grammar of section~\ref{section:syntax} does not require type information for disambiguation.
Hence, the symbol table only keeps track of identifiers in scope and their associated sort without types.

The following sections present alternative production rules used in place of those in the grammar of section~\ref{section:syntax} obtained by merging nonterminals that lead to ambiguities.
Rewriting the grammar for optimization, left recursion elimination, and stateless ambiguity resolution with recursive descent is outside the scope of this presentation.

\subsection{Resolving Syntactic Ambiguities for \acs{LF}'s Grammar}\label{section:blurred-lf-syntax}

The grammar for \LF kinds, types and terms from section~\ref{section:syntax-lf} is data-dependent because \LF term variables, type constants and term constants are syntactically indistinguishable from one another, and likewise for $ \Pi $-kinds and $ \Pi $-types.
This next grammar blurs together \LF kinds, types and terms with corresponding nonterminals \synt{lf-kind}, \synt{lf-type} and \synt{lf-term} for context-free parsing.
Occurrences of \synt{lf-kind}, \synt{lf-type} and \synt{lf-term} elsewhere in the grammar are replaced with \synt{lf-object}.
Disambiguation of an \LF object as presented below requires:
\begin{itemize}
\item a referencing environment to resolve constants and variables, and
\item a target sort for the elaboration (i.e. knowing beforehand whether the \LF object should be a kind, type or term).
\end{itemize}

\begin{grammar}
<lf-object> ::= <identifier>
\alt <qualified-identifier>
\alt `type'
\alt `_'
\alt `{' <omittable-identifier> [`:' <lf-object>] `}' <lf-object>
\alt `\\' `(' <omittable-identifier> `:' <lf-object> `)' `.' <lf-object>
\alt `\\' <omittable-identifier> `.' <lf-object>
\alt <lf-object> <forward-arrow> <lf-object>
\alt <lf-object> <backward-arrow> <lf-object>
\alt <lf-object> `:' <lf-object>
\alt <lf-object> <lf-object>
\alt `(' <lf-object> `)'
\end{grammar}

Additionally, the syntax is disambiguated in order of decreasing precedence as follows:

\begin{enumerate}
\item The juxtaposition of \LF objects (separated by whitespace to denote application) is left-associative.
\item User-declared infix, prefix (right-associative) and postfix (left-associative) operators.
\item Forward arrows are right-associative, and backward arrows are left-associative, with equal precedence, hence they may not both appear at the same precedence level.
\item Type ascriptions are left-associative.
\item Binders are weak prefix operators, meaning that the identifier they introduce is in scope for the entire \LF object on the right.
\end{enumerate}

\subsection{Resolving Syntactic Ambiguities for Contextual \acs{LF}'s Grammar}

The grammar for contextual \LF types, terms, patterns, substitutions, contexts and their patterns from section~\ref{section:syntax-contextual-lf} is data-dependent because contextual \LF term variables, type constants and term constants are syntactically indistinguishable from one another.
This next grammar blurs together contextual \LF types, terms and patterns with corresponding nonterminals \synt{clf-type}, \synt{clf-term} and \synt{clf-pattern} for context-free parsing.
Occurrences of \synt{clf-type}, \synt{clf-term} and \synt{clf-pattern} elsewhere in the grammar are replaced with \synt{clf-object}.
Likewise, the nonterminals \synt{clf-substitution}, \synt{clf-substitution-pattern}, \synt{clf-context} and \synt{clf-context-pattern} are blurred together as \synt{clf-context-object}.
Disambiguation of a contextual \LF object as presented below requires:
\begin{itemize}
\item a referencing environment to resolve constants and variables, and
\item a target sort for the elaboration (i.e. knowing beforehand whether the contextual \LF object should be a type, term or pattern).
\end{itemize}
Additionally, contextual \LF contexts starting with an untyped identifier are always disambiguated as having a context variable in head position.

\begin{grammar}
<clf-object> ::= <identifier>
\alt <hash-identifier>
\alt <dollar-identifier>
\alt <qualified-identifier>
\alt `{' <omittable-identifier> [`:' <clf-object>] `}' <clf-object>
\alt `\\' `(' <omittable-identifier> `:' <clf-object> `)' `.' <clf-object>
\alt `\\' <omittable-identifier> `.' <clf-object>
\alt <clf-object> <forward-arrow> <clf-object>
\alt <clf-object> <backward-arrow> <clf-object>
\alt `block' `(' [<identifier> `:'] <clf-object> (`,' [<identifier> `:'] <clf-object>)* `)'
\alt `block' [<identifier> `:'] <clf-object> (`,' [<identifier> `:'] <clf-object>)*
\alt <clf-object> `:' <clf-object>
\alt <clf-object> <clf-object>
\alt `_'
\alt `?'[<identifier>]
\alt <clf-object> `[' <clf-context-object> `]'
\alt `<' <clf-object> (`;' <clf-object>)* `>'
\alt <clf-object> <dot-integer>
\alt <clf-object> <dot-identifier>
\alt `(' <clf-object> `)'

<clf-context-object> ::= [`^']
\alt `..'
\alt [`..' `,'] [<identifier> `:'] <clf-object> (`,' [<identifier> `:'] <clf-object>)*
\end{grammar}

Additionally, the syntax is disambiguated in order of decreasing precedence as follows:

\begin{enumerate}
\item The juxtaposition of contextual \LF objects (separated by whitespace to denote application) is left-associative.
\item User-declared infix, prefix and postfix operators.
\item Projections are left-associative.
\item Substitutions are left-associative.
\item Forward arrows are right-associative, and backward arrows are left-associative, with equal precedence, hence they may not both appear at the same precedence level.
\item Type ascriptions are left-associative.
\item Binders are weak prefix operators, meaning that the identifier they introduce is in scope for the entire contextual \LF object on the right.
\end{enumerate}

\subsection{Resolving Syntactic Ambiguities for the Meta-Level's Grammar}

The grammar for context schemas, meta-types, meta-objects, meta-patterns and meta-contexts from section~\ref{section:syntax-meta-level} is ambiguous because the contextual \LF types and terms therein have ambiguous grammars.
This next grammar blurs together meta-types, meta-objects and meta-patterns with corresponding nonterminals \synt{meta-type}, \synt{meta-object} and \synt{meta-pattern} for context-free parsing.
Occurrences of \synt{meta-type}, \synt{meta-object} and \synt{meta-pattern} elsewhere in the grammar are replaced with \synt{meta-thing}.
Likewise, the nonterminals \synt{schema} and \synt{meta-context} are replaced with \synt{schema-object} and \synt{meta-context-object} respectively.
Disambiguation of a meta-thing as presented below requires:
\begin{itemize}
\item a referencing environment to resolve constants and variables, and
\item a target sort for the elaboration (i.e. knowing beforehand whether the meta-thing should be a meta-type, meta-object or pattern).
\end{itemize}

\begin{grammar}
<schema-object> ::= <qualified-identifier>
\alt <schema-object> `+' <schema-object>
\alt [`some' `[' <identifier> `:' <lf-object> (`,' <identifier> `:' <lf-object>)* `]'] `block' `(' [<identifier> `:'] <lf-object> (`,' [<identifier> `:'] <lf-object>)* `)'
\alt [`some' `[' <identifier> `:' <lf-object> (`,' <identifier> `:' <lf-object>)* `]'] `block' [<identifier> `:'] <lf-object> (`,' [<identifier> `:'] <lf-object>)*

<meta-thing> ::= <qualified-identifier>
\alt `(' <clf-context-object> <turnstile> <clf-context-object> `)'
\alt `#(' <clf-context-object> <turnstile> <clf-context-object> `)'
\alt `$(' <clf-context-object> <turnstile> <clf-context-object> `)'
\alt `$(' <clf-context-object> <turnstile-hash> <clf-context-object> `)'
\alt `[' <clf-context-object> `]'
\alt `[' <clf-context-object> <turnstile> <clf-context-object> `]'
\alt `#[' <clf-context-object> <turnstile> <clf-context-object> `]'
\alt `$[' <clf-context-object> <turnstile> <clf-context-object> `]'
\alt `$[' <clf-context-object> <turnstile-hash> <clf-context-object> `]'

<meta-context-object> ::= [`^']
\alt <meta-object-identifier> `:' <meta-thing> (`,' <meta-object-identifier> `:' <meta-thing>)*
\end{grammar}

Additionally, in a meta-thing, a fully qualified identifier referring to a context schema constant is ambiguous with a context variable.
Such context schema constants are disambiguated from the parsed context object.

\subsection{Resolving Syntactic Ambiguities in Computations}

The grammar for computation-level kinds and types from section~\ref{section:syntax-computations} is data-dependent because computational base and cobase type constants are syntactically indistinguishable from one another, and because the grammar relies on meta-level objects.
This next grammar blurs together computational kinds and types with corresponding nonterminals \synt{comp-kind} and \synt{comp-type} for context-free parsing.
Occurrences of \synt{comp-kind} and \synt{comp-type} elsewhere in the grammar are replaced with \synt{comp-sort-object}.
Disambiguation of a computational sort object as presented below requires:
\begin{itemize}
\item a referencing environment to resolve constants and variables, and
\item a target sort for the elaboration (i.e. knowing beforehand whether the computation-level sort object should be a kind or a type).
\end{itemize}

\begin{grammar}
<comp-sort-object> ::= <identifier>
\alt <qualified-identifier>
\alt `ctype'
\alt `{' <omittable-meta-object-identifier> [`:' <meta-thing>] `}' <comp-sort-object>
\alt `(' <omittable-meta-object-identifier> [`:' <meta-thing>] `)' <comp-sort-object>
\alt <comp-sort-object> <forward-arrow> <comp-sort-object>
\alt <comp-sort-object> <backward-arrow> <comp-sort-object>
\alt <comp-sort-object> `*' <comp-sort-object>
\alt <comp-sort-object> <comp-sort-object>
\alt <meta-thing>
\alt `(' <comp-sort-object> `)'
\end{grammar}

Additionally, the syntax is disambiguated in order of decreasing precedence as follows:

\begin{enumerate}
\item The juxtaposition of computational sort objects (separated by whitespace to denote application) is left-associative.
\item User-declared infix, prefix (right-associative) and postfix (left-associative) operators.
\item Forward arrows are right-associative, and backward arrows are left-associative, with equal precedence, hence they may not both appear at the same precedence level.
\item Binders are weak prefix operators, meaning that the identifier they introduce is in scope for the entire sort object on the right.
\end{enumerate}
